\begin{appendices}

\chapter{Self-appraisal}

    \section{Critical self-evaluation}

        % a bit ambitious
        The topic of this project is such that its scope can be very large. As
        mentioned in the main text (\autoref{sec:methodology}), proper
        implementations are thousands of lines of code written over many years
        by lots of people. As such, this project is very ambitious and choosing
        it was risky. The quantity of new things that had to be gone over was
        immense and could have easily led to a failed implementation.
        Nevertheless, I am glad to have taken the challenge as it has produced
        a much more interesting dissertation. Having a simpler topic would have
        probably meant that a higher grade would have been easier to achieve,
        although that would have been unlikely to happen as the motivation for
        a less interesting project would have probably lacked.

        % heavy on research
        To that effect, this project is a bit heavy on research. Even though I
        knew a lot about filesystems, in order to get a good grasp of things
        the research started way back with UFS. It is the first relevant
        filesystem but it is also over 50 years old and a lot has happened
        since then. The idea was to go over the major developments over the
        years and then focus on what is modern. This led to many more topics
        that I had not considered before and as a result the project considers
        things it would not have otherwise. I do not believe that skipping a
        lot of this old research would have made a better project though. There
        was limited time not everything could be covered unfortunately. If more
        time was available though, I would only look at the most recent
        research to make it more rigorous.

        % feature selection

        Fully going through such an ambitious project was definitely not an
        option, however. That meant that narrowing the scope down was a big
        part of the journey. The breadth of possible features meant most things
        that make up a filesystem could not make it to the final product and
        deciding what to cut and what to keep was not easy. In most cases,
        decisions were a compromise between many factors and what was settled
        on was the right choice for this project. The only exception is the
        choice to drop extents. From a complexity standpoint, dropping it was
        the right choice to complete the project on time. However, it turned
        out to be a poor choice for performance. This was discovered very late
        and changing it was not an option. As a result, the filesystem is not
        usable for real world tasks in this state, but I do believe that even
        with this major flaw, this is a valuable experiment and can open the
        way for further research.

        % thank fuck for the process
        To properly do these decisions, the adopted process was a blessing
        considering how much could not be included. The ambitious topic meant
        that there was a lot to consider. Not having a fixed arbitrary goal
        (say "a performant filesystem") meant the project could change a lot.
        In fact it changed direction quite a bit during research and its
        implementation took several turns. Is still not complete but I do
        believe it is the best it could be.

        % infinite time, stubborn
        In the end, implementing the project and writing the report went well
        overall. Progress on the implementation went at a steady pace and was
        interleaved with lots of research to make sure it is correct. When the
        report was started it integrated a lot of the syntax necessary to save
        work later. However, in both cases a sense of infinite time was a big
        problems. This being a final project meant that the desire for it to be
        as good as possible overshadowed any consideration for how long this
        would reasonably take. When the scope settled in, motivation quickly
        dried up which became a problem. Time turned out to be very much
        limited and things had to be cut short or reconsidered. This was most
        painfully apparent when it came time to editing the report. Since I had
        experience with big MS Word documents before I was confident the \LaTeX{}
        version would be quicker as it provides many features almost out of the
        box provided some guidelines are obeyed. However, \LaTeX{} sets the bar
        much higher than Word and as a result graphics suffer. If less
        stubborn confidence was put on \LaTeX{} the report might have looked even
        better.

    \section{Personal reflection and lessons learned}

        My biggest lesson to be learnt is the importance of progress tracking.
        Having a record of what has happened when is very motivating as it
        shows the finish line getting closer. Even a simple piece of paper with
        bullet points goes a long way. Not having this meant that getting stuck
        debugging a problem felt like wasted time. Even though after it was
        done the project was in much better shape and much closer to being
        "good", doing it felt pointless and ending days abruptly was a
        problem.

        Another thing is to not stick with decisions when there is enough
        evidence they might need to be reconsidered. In particular, the linked
        list for data blocks (\autoref{sec:design_btree}) felt "almost done"
        for several weeks. A lot of time was sunk into trying to make it better
        when it was ultimately meant to be scrapped and replaced with a B-tree.
        Moving up that work would have been the reasonable thing to do
        especially since the chosen methodology (\autoref{sec:methodology})
        meant this was easy to do.

        Finally, the constant aim for progress means things I am familiar
        things should not be underestimated and deprioritised. Particularly,
        since I felt more confident with \LaTeX{}, I thought less time would be
        necessary to make it work. This was true in that to achieve simple
        things was quicker, but the bar was much higher now, so at the end of
        the day it took just as much time just to do fancier things that I
        never though would be included.

    \section{Legal, social, ethical and professional issues}

        \subsection{Legal issues}
            \label{app:licence}

            This project has one legal issue. It borrows code from the Linux
            kernel which is licensed under the GNU General Public License
            version 2 (GPL v2). This license is restrictive to the extent that
            it mandates any modification of borrowed code to be released under
            the terms of the same license. To comply with this requirement, the
            borrowed B-tree code maintains all license and copyright notices.
            Further, the whole project adopts the GPLv2. This eliminates the
            chance for any licence conflict and makes sure the new code stays
            public.

        \subsection{Social issues}

            As a piece of infrastructure this project does not relate or
            benefit other disciplines such as education. It also does not
            further issues of crime, bullying, privacy or (in)equality. It
            cannot be used to facilitate criminal activity insofar as any data
            storage system can. The project is open source and available to all
            regardless of skill, social or cultural background. As such it is a
            benefit to society free of charge and with no usage limitation as
            the GPLv2 guarantees that. Finally, it aims to make data integrity
            easier to deploy allowing to easy some of society's everyday
            problems.

        \subsection{Ethical issues}

            This project is not affected by ethical issues. That is becauase it
            does not deal with human subjects directly and where it is
            indirectly involved with humans, it provides no new tools where an
            ethical issue might arise. Particularly, this project does not deal
            with the particularly sensitive field of artificial intelligence
            and its use of cryptography is limited to hashing. Any possible use
            of the project is already covered by existing filesystems such as
            ext4 in regards to efficiency or ZFS for reliability. Where
            features that ZFS provides can be used for criminal and morally
            dubious activity, like data integrity, are meant to be a weaker in
            this project from the project definition (\autoref{sec:problem}).
            Where features that ext4 provides can be used for similar purposes,
            like large scale deployments, this filesystem purposefully scales
            back to provide redundancy. As such this is an established
            technology with known good ethical implication.

        \subsection{Professional issues}

            This project has been developed in accordance with the principles
            of the British Computer Society's (BSC) code of conduct. In
            particular, it aims to further the public interest in that it
            provides a useful tool for home use. It uses a licence of good
            standing so its source can be freely available and inspected for
            security and standards compliance.

            The project uses standard practices, widely used frameworks and
            does not rely on proprietary bits to hide parts of itself. It is
            transparent and upfront about its purpose and capabilities.

\chapter{External Material}

    \begin{enumerate}
        \item B-trees from Linux (lib/btree.c)
        \item Linux's FUSE framework
    \end{enumerate}

\chapter{Raw benchmark output}
    \label{app:benchmark}

    This appendix shows the full output of the fio command for one of the three
    runs on each tested filesystem.

    \section{Results for ext4}
        \lstinputlisting[language={},basicstyle=\tiny]{benchmark/benchmark.ext1}
    \section{Results for ntfs}
        \lstinputlisting[language={},basicstyle=\tiny]{benchmark/benchmark.ntfs1}
    \section{Results for the proposed filesystem}
        \lstinputlisting[language={},basicstyle=\tiny]{benchmark/benchmark.own1}

\chapter{perf outputs}
    \label{app:perf}

    \section{\monospace{perf} report with overhead of children added to their parents}
        \lstinputlisting[language={},basicstyle=\tiny]{benchmark/perf.hist.0}
    \section{\monospace{perf} report only for time speant in a subroutine itself}
        \lstinputlisting[language={},basicstyle=\tiny]{benchmark/perf.hist.1}

\end{appendices}
