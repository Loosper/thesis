\begin{appendices}

%
% The first appendix must be "Self-appraisal".
%
\chapter{Self-appraisal}

    % <This appendix should contain everything covered by the 'self-appraisal'
    % criterion in the mark scheme. Although there is no length limit for this
    % section, 2---4 pages will normally be sufficient. The format of this
    % section is not prescribed, but you may like to organise your discussion
    % into the following sections and subsections.>

    \section{Critical self-evaluation}

    \section{Personal reflection and lessons learned}

    \section{Legal, social, ethical and professional issues}

    % <Refer to each of these issues in turn. If one or more is not relevant to
    % your project, you should still explain {\em why} you think it was not
    % relevant.>

        % TODO ask should i keep copyright?
        \subsection{Legal issues}
            \label{app:licence}

            This project has one legal issue. It borrows code from the Linux
            kernel which is licensed undre the GNU General Public License
            version 2 (GPL v2). This license is restrictive to the extent that
            it mandates any modification of borrowed code to be released under
            the terms of the same license. To comply with this requrement, the
            borrowed B-tree code maintains all license and copyright notices.
            Further, the whole project adopts the GPL v2.


        \subsection{Social issues}

            The project does not address nor does it further any social issues.
            As a piece of infrastructure it does not relate or benefit other
            disciplines such as education. It also does not further issues of
            crime, bullying, privacy or (in)equality. The project is open
            source and available to all regardless of skill, social or cultural
            background. It cannot be used to facilitate criminal activity
            insofar as any data storage system can.

        \subsection{Ethical issues}

            This project is not affected by ethical issues. That is becuase it
            does not deal with human subjects directly and where it is
            indirectly involved with humans, it provides no new tools where an
            ethical issue might arise. Particularly, this project does not deal
            with the particularly sensistive field of artificial intelligence
            and its use of cryptography is limited to hashing. Any possible use
            of the project is already covered by exisiting filesystems such as
            EXT4 in regards to efficiency or ZFS for reliability. Where
            features that ZFS provides can be used for criminal and morally
            dubous activity, like data integrity, are meant to be a weaker in
            this project from the project definition (\autoref{sec:problem}). Where
            features that EXT4 provides can be used for similar purposes, like
            large scale deployments, this filesystem purposefully scales back
            on to provide redundancy.

        \subsection{Professional issues}

            This project has been developed in accordance with the principles
            of the British Computer Society's (BSC) code of conduct. In
            particular, it aims to be further the public interest in that it
            provides a useful tool for home use...
            % TODO:


%
% Any other appendices you wish to use should come after "Self-appraisal". You can have as many appendices as you like.
%
\chapter{External Material}

    % <This appendix should provide a brief record of materials used in the
    % solution that are not the student's own work. Such materials might be
    % pieces of codes made available from a research group/company or from the
    % internet, datasets prepared by external users or any preliminary
    % materials/drafts/notes provided by a supervisor. It should be clear what
    % was used as ready-made components and what was developed as part of the
    % project. This appendix should be included even if no external materials
    % were used, in which case a statement to that effect is all that is
    % required.>

    \begin{enumerate}
        \item B-trees from Linux
    \end{enumerate}

\end{appendices}
