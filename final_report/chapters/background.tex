% You can cite chapters by using '\ref{chapter1}', where the label must
% match that given in the 'label' command, as on the next line.
\label{background}
% Sections and sub-sections can be declared using \section and \subsection.
% There is also a \subsubsection, but consider carefully if you really need
% so many layers of section structure.
\chapter{Introduction and Background Research}

    \section{Introduction}

    % <A brief introduction suitable for a non-specialist, {\em i.e.} without
    % using technical terms or jargon, as far as possible. This may be
    % similar/the same as that in the 'Outline and Plan' document. The
    % remainder of this chapter will normally cover everything to be assessed
    % under the `Background Research` criterion in the mark scheme.>

    % Here is an example of how to cite a reference using the numeric system
    % \cite{parikh1980adaptive}.


    % Must provide evidence of a literature review. Use sections
    % and subsections as they make sense for your project.

% TODO: this is a temporary name
\chapter{Lit review}
    \section{The UNIX filesystem}
        \label{UFS}

        For the past 50 years the UNIX style operating system has been one of
        the most prevalent operating system architectures out there. The
        original UNIX brought about a simple and elegant template which is
        still with us to this day. It brough forward many novel ideas: wide
        hardware compatibility, simple and powerful system calls, a powerfull
        shell and minimal design and most importantly a file system
        \cite{UNIX}. All these features are still with us, largely unmodified,
        to this day, although greatly expanded and improved upon.

        The most important part, the filesystem, had an implementation akin to
        its interface - simple and elegant. It introduced several important
        concepts:

        From a programming view, a file is a linear sequence of bytes and can
        be controlled with just 4 system calls: \syscall{open} \syscall{read},
        \syscall{write} and \syscall{close}. No structure is enforced on it
        from the operating system. A file can be \syscall{open}ed by name to
        get a handle to it. Programs can \syscall{read} all of its contents and
        \syscall{write} back anything new. The operating system maintains an
        offset showing the current location within the file, so sequential
        calls to \syscall{read} and \syscall{write} begin operation from that
        location and increment it by the amount they modified. No limit is
        imposed on file size, so writing past the end of the file increases it.
        A file must be \syscall{close}ed after use.

        From an implementation view, every file has an i-number. This is a
        number that uniquely identifies this file, regardless of its path or
        location. It is essentially the pointer to the file in the table of
        files, called an i-list (or i-table more recently).

        Each entry in the i-list is an i-node (modernly shortened to just
        inode). The inode is essentially the header for the file itself, it has
        all metadata including pointers to the actual data.

        To make this flat list of files into a usable tree, special files are
        introduced, called directories, marked by a special bit in the inode.
        The directory is a file whose contents have a special format: a list of
        2-tuples (name, i-number), which denote that file number
        \textit{i-number} can be found in this directory under name
        \textit{name}. Directories are special in that they cannod be read and
        written directly, rather special system calls must be used for the
        operating system to report its contents. The exact method is slightly
        involved and not relevant for this dissertation, but it is sufficent to
        say that there are simple library functions roughly equvalent to those
        for normal files.

        File names are alphanumeric strings of any length. To denote that a
        file is contained within a directory, a single forward slash ("/") is
        used, meaning whatever is on the right of it is contained on the
        directory named on the left. This can be repeated an unlimited number
        of times. The root directory has no name \textit{per se}, hence any
        path refering to it starts with a slash, for example
        \textit{/file/in/root}.

        File number 1 is defined to be the the root directory to bootrstrap the
        search and inode 0 does not exist.

        An important, but not user visible detail (in any way) is that the
        filesystem has a list of free blocks (called the free list), to denote
        which space is unused by the system, where a block is a fixed chunk of
        space on disk.


    \section{Evolution of implementation}

        \subsection{UFS}

            Initially, this model had a very simple implementation. The
            original UNIX used a linear, fixed size i-list. An inode had 13
            pointers to data blocks, 10 of which point directly to data blocks,
            while the next 3 have 1, 2 and 3 levels of indirection
            respectively. The free list was a simple linked list, where
            each block had pointers to free blocks and a pointer to the
            next block in the list \cite{UNIX_implementation},
            \cite{UNIX_thompson}.

            Over time, however, it turned out that this arrangement was very
            inefficient. In particular, data blocks for a file lacked locality,
            as did the i-numbers for files in the same directory. Additionally,
            the free list would become very scrambled in that sequential
            entries in the list may be wildly far apart on disk. All of these
            problems got exacerbated after prologned use. On top of all that,
            due to hard drives (the main storage medium) being a physical
            spinning platter with a head that has to move to the track on drive
            and wait for the block to come under it (operation knows as
            "seek"), seek times for arbitrary blocks was large. As a result,
            throughput could be as low as 4\% \cite{FFS}.


        \subsection{FFS}

            % maybe mention that increasing block size decreases the need for
            % indirection
            \citeauthor{FFS} proposed many improvements to improve performance.
            For a start, block size was increased 8 fold from 512 bytes to 4096
            (or even 8192), which improved locality and reduced seek times,
            essentially for free, although extra complexity was introduced to
            minimise wasted space from small files. Further, blocks were
            grouped together into cylinder groups, where a cylinder is a
            physical feature of a hard drive, blocks on which have very good
            sequential performance.  Then, policies were introduced to keep
            data blocks of a file and inodes in a directoy sequential and on
            the same cylinder group where possible. Finally, the free list
            became a bitmap of a fixed size, eliminating the scrambling of
            supposedly sequential entries that further degrade performance.

            The result was that \citeauthor{FFS} measured an increase in disk
            bandwidth utilisation of about an order of magnitude without an
            increase in wasted space.

            % TODO: do i need this
            % An important note is that cylinder groups include redundant copies
            % of various metadata to increase resiliance to data loss from damage
            % to the disk.

            % Another important note is that filesystems which are near their
            % capacity suffer greatly decreased performance.

        \subsection{LFS}

            With advances in software and hardware, by \citeyear{IO_bottleneck}
            file access and reliability patterns became apparent. Especially
            for office use, small reads and writes dominate the workload. The
            physical nature of spinning hard drives make such operations costly
            when performed in a random manner. To imporove performance, caching
            is identified as a substantial improvement, which has been wildly
            used since \cite{Linux_caching}, \cite{IO_bottleneck}.


            However, \citeauthor{IO_bottleneck} points out some issues with
            such an approach: a system crash or a power outage puts the
            contents of such a cache in jeopardy. This is not a problem for a
            read cache, but a write cache loss can be catastraphic. This brings
            reliability into the picture and balancing performance for
            reliability becomes a key factor.

            Further, \citeauthor{IO_bottleneck} argues that such caching shifts
            the IO pattern from mostly random reads to mostly writes in
            occasional big bursts of relatively large amount of data.
            Maintaining the usual layout of filesystems popular so far does not
            make use of this fact, so the author suggests structuring the
            filesystem as a log - an append-only structure which wraps around
            once it fills up. This gives locality to files that are used
            together, utilises the burst nature of writes and recovery is easy
            - only the end may be corrupted \cite{IO_bottleneck}.

            It must be noted that maintaining a fraction of the disk free is
            once again emphasised as critical for good performance.

            \citeauthor{LFS}, reiterates the necessity of asynchronous writes
            (and therefore a cache). In LFS \cite{LFS}, based on
            \cite{IO_bottleneck}, the inode structure is kept exactly the same
            as in FFS (including the data block arrangement), but its layout
            policy was redically different. First, the i-list is now a regular
            file, but only contains addresses to where the actual inode can be
            found. The inode itself and its data are laid our sequentially to
            improve performance and the free list is done away with. Instead,
            the disk is split into large chunks of free continuous space -
            \textbf{extents} (called segments in LFS) - which contain a single
            block of metadata that includes that information.  Special care is
            taken to maintain these chunks large.  This is done to reduce
            fragmentation and to allow a better likelyhood for sequential
            accesses. It is important to note, that the idea for extents
            remains present to this day.

            New data is written in a log fashion. There are several fixed
            places where a pointer to the i-list is stored and there is a
            single fixed place that is dedicated to storing the last known good
            head of the log. This ensures reliability.

        \subsection{WAFL}

            With time reliability becomes an ever growing concern, both from a
            techincal and from a human perspective. The idea of snapshots
            contributes to both of these.

            Snapshots are read-only copies of the entire file system. Many of
            them are kept, allowing for old versions of files to be viewed in
            case of deletion. WAFL uses copy-on-write for disk blocks to avoid
            duplicating data \cite{WAFL}. Snapshots are essentially pointers to
            these blocks, hence a new snapshot consumes space only after data
            is modified. Block are freeded after all references to them are
            freed too.

            Besides file versioning, snapshots are useful for backups on the
            live system, since they are pretty much free on WAFL.


            The way WAFL implemnts the copy-on-write is by creating a tree of
            blocks, where the root is a known fixed block. Then it has all
            metadata (like i-list and free list) be a file too. The inodes of
            all of these files are stored in an inode file, whole inode itself
            is stored in the root block. This way every single bit of info
            required for the filesystems operation is connected in a rooted
            tree and copying the root block makes a snapshot. Modifying any
            other block requires copying to a new location with the
            modificaiton only on the new copy and modifying the reference to
            it. Modifying the reference requires the same operation, which
            eventually bubbles up to the root block which can be modified in
            place.

            The benefit of this approach is that the filesystem is always
            consistent. At no point is any data modified in place, meaning a
            crash cannot cause corruption. Rather a new copy is built somewhere
            in free space and only "added" to the filesystem when the whole
            branch of the tree is complete. Writing a single block is assumed
            to be atomic \cite{???}, so updating the root an all or nothing
            operation, limiting the impact of a crash to only some data loss
            that hasn't had the time to be commited to disk.

    \section{EXT4}

        The most modern version of the ext filesystem, ext4, is native and
        usually default on Linux. It integrates a lot of the improvements over
        the years while maintaining the same general idea. Since it was
        developed for (and by) Linux, which is a UNIX derivative, the ext
        family follows closely the same principles present over 50 years ago in
        UFS (section \ref{UFS}).

        It is a good benchmark for modern filesystems as it is the default on
        many (if not most) major Linux ditributions
        \cite{https://wiki.debian.org/FileSystem},
        \cite{https://access.redhat.com/documentation/en-us/red_hat_enterprise_linux/6/html/performance_tuning_guide/s-storage-fs}.
        It is actively maintained
        \cite{https://www.spinics.net/lists/linux-ext4/} and very mature, first
        appearing in version 2.6.19 in 2006
        \cite{http://www.h-online.com/open/features/Kernel-Log-Higher-and-Further-The-innovations-of-Linux-2-6-28-746805.html}.
        Finally, unlike its most direct competitiors by the likes of Windows'
        NTFS and MacOS' APFS, it is free and open source, so code and
        documentation are readily available.
    \section{Data integrity}

        % TODO: validate from paper, becase most of this is my general
        % knowledge (espcially parity calculation)
        \subsection{RAID}

            Non-volatile storage mediums have historically been orders of
            magnitude slower than volatile memories like RAM which are
            themselves much slower than CPUs and their caches \cite{raid has a
            Frank87, Stevens81}. Disk arrays are an attempt to increase IO
            performance and bridge that gap. However, drives are inherently
            unreliable \cite{pls (backblaze is a killer source)}, so putting
            many of them together exacerbates the problem \cite{maybe also
            raid. he has an MTTF caluclation}.  RAID is one solution for the
            problem of drive reliability \cite{RAID}. As its name suggests,
            RIAD adds redundancy to disk arrays. Popular RAID arrangements are
            RAID 1 and RAID 4/5.

            RAID 1 is two drives acting as one, effectively a mirror. If one of
            them fails the other is a perfect copy.

            RAIDs 4 and 5 are very similar. Both calculate extra parity for
            each block (by XORing the respective blocks on all disks together)
            and store it on an extra disk. The major difference is how they lay
            out parity across the disks. The result is that the array can lose
            a disk any of the disks while keeping all of the data intact. It
            has a benefit over RAID 1 in that it is cheaper.

            Even though not part of the original paper, RAID 6 is also very
            common. It is essentially RAID 5 but with more copies of the parity
            data (equal to the number of parity disks).

        \subsection{drive reliability}

            Drives are unreliable \cite{RAID}, \cite{Backblaze (idk which one)}.
            Due to their mechanical nature they wear out and stop working. They
            do this at an unpredictable rate for individual drives, but at
            relatively constant rates over large arrays \cite{Backblaze_stats}.
            Tools for monitoring drive health, like SMART, have been developed
            to combat the issue and are in general a good indicator for the
            likelyhood of loss of the drive.

            However, drive failure is not binary. They do not work flawlessly
            until such a time as to not work at all \cite{2D_RAID}. Individual
            sectors develop errors over time, failing to be read or written to.
            Drives try to combat the issue by keeping an array of spare sectors
            that replace misbehvaing ones when detected. To determine if a
            sector is misbehvaing in the first place, drives employ ECC codes
            \cite{data_corruption_storage_stack, although maybe someone else}.

            However, despite all the best efforts of drives and their
            manufacturers, drives still cannot be relied upon to report correct
            data all of the time \cite{data_corruption_storage_stack}.
            Occasionally, they will report an unrecoverable error to the
            operating system. Even more rarely, drives will return incorrect
            data without indicating errors. As a consequence, individual drives
            are never relied upon for the data to be present and intact and
            other techniques are always employed to ensure both of these
            independently \cite{LTT_data_loss?}.

            SSDs and flash storage in general are a similar tale. As
            \citeauthor{2D_RAID} puts it: "Flash is basically garbage. It
            barely works. We tolerate it because its fast" \cite{2D_RAID}. They
            employ similar techniques as spinning hard drives and suffer
            similar problems as them \cite{flash_large_scale},
            \cite{flash_reliability}. Their failure patterns are substantially
            different but the end result is the same - unrecoverable errors
            that would lead to data loss if not accounted for outside the
            drive.

        \subsection{At scale}

            A unique case for data reliability are datacenters for which data
            loss is not an option. They also operate at massive scales, where
            conventional approaches break down or become severe bottlenecks.
            Hence many, if not most, of these companies have special bespoke
            arrangements to fulfil their needs.

            \subsubsection{Backblaze}

                Conveniently, Backblaze, a cloud storage provider, has an open
                approach to its infrastructure, frequently publishing in-depth
                information and open sourcing parts of their infrastructure.

                Their solution \cite{Backblaze_arch} to reliability is
                conceptually distributed RAID 6 over 20 drives, 3 of which are
                parity. However, they do not use conventional RAID. Instead,
                they use custom software to split files into chunks they call
                "shards" which are then stored on normal ext4 filesystems. The
                choice is motivated by the good reliability and performance of
                ext4 and that inverting the usual order of replication and
                filesystem allows them to protect against filesystem
                corruption.

                Additionally, every shard is checksummed to protect against
                data corruption. This is necessary because "once in a while
                they return the wrong data, or are just unable to read a given
                sector" \cite{Backblaze_arch} and software needs to be able to
                identify the error and recompute the data from the parity.

