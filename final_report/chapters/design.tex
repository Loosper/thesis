\chapter{Design}

    \section{Methodology}
        \label{sec_methodology}

        % TODO: (some) duplicate with discussion
        Designing a filesystem is difficult and implemneting it even more so.
        For example, EXT4's documentation \cite{ext4_docs} is roughly 20000
        words of mostly major design features, and around 60000 lines of Linux
        kernel code. For this reason designing and implemneting this filesystem
        followed a peculiar process to reduce the risk of having an incomplete
        prototype by the submission deadline due to comlextiy.

        The process boils down a single principle: always have a working
        prototype. The idea is to start with the smallest and simplest
        implementation that works and then add individual features
        incrementally in a manner that always ends up with a working prototype.
        The plan was to initially complete the large amount of work to have
        \textit{any} filesystem and add geatures until time allowed. This way,
        no matter the unexpected obstacles in writing a filesystem and
        underestimation of effort turn out to be, there is always a submittable
        project to get some (even if lower) marks.

        % TODO: you can note that the kernel uses a similar thing with their patch
        % process. Think how I developed the arm internship code

    % TODO: is this necessary
    \section{Design goals}

        The backdrop for the design that follows are two goals: demonstrating
        integrity guarantees in a complete filesystem and maintaining
        simplicity of the design. The main reason behind them is the limited
        time and large potential scope of the project. It is intended that this
        project only be a prototype and not production ready infrastructure.
        This means avoiding special purpose data structures and sticking to the
        most basic solution that gives satisfactory results.

    \section{Metadata as files}
        \label{sec_files}

        The first design choice was that all metadata in the filesystem would
        be kept as a regular file instead of having special data structures for
        each, heavily inspired by WAFL (\ref{sec_WAFL}), although EXT4
        (\ref{sec_EXT4} also stores a substantial amount of metadata (but not
        all) as regular files too. This way they can all benefit from the
        underlying indexing method and substantially reduce implementation
        complexity. Further, this automatically solves the problem of how to
        allocate space for them.  For example, since the inode table is a file,
        there is no need make any special consideration for its initial size.
        It can grow as necessary, get spread throughput the filesystem as
        available and this approach will accommodate both edge cases equally
        well. On one end, where the filesystem has to store very few very large
        files, no space will be wasted towards an unused i-list. On the other
        end, where it has to store a very large amount of minuscule files, the
        ilist can grow to a substantial fraction of all storage, without
        running into an arbitrary limit.

    \section{Block size}
        \label{sec_block_size}

        % TODO: cite the all claim?
        As \citeauthor{FFS} explains (see \ref{sec_FFS}), larger block sizes
        improve throughput and reduce overhead, even on SSDs (see
        \ref{sec_hardware}). As a sensible default blocks are set to 4KiB in
        size, as viertually all filesystems do. It offers a good tradeoff
        between maximising IO performance and minimising overhead for things
        like free lists while still being small enough to not waste space if
        unused in small files.

        Extents (\ref{sec_LFS}) are a popular feature to increase throughput.
        This filesystem does not include them as they significantly increase
        complexity and their benefits only become apparent when combined with
        other features such as tree based free space tracking (described in
        \ref{sec_free_list}), more involved allocation algorithms like those in
        EXT4 (\ref{sec_EXT4}) or those that \citeauthor{ext4_space_maps}
        describes.

    \section{The free list}
        \label{sec_free_list}

        A popular way keep track of available free space is block bitmaps.
        Historically, many filesystems have frequently used it like FFS
        \cite{FFS}, WAFL \cite{WAFL} and more recently, EXT4
        \cite{ext4_space_maps} and HFS+ \cite{HFSplus}.  They are very space
        efficient, using only 256 KB per GB with 4KB blocks or about 0.02\%.
        Their small size allows them to be easily cached and kept up to date. A
        major drawback is that traversing these bitmaps is not particularly
        efficient as it usually devolves to some form of linear search,
        although some techniques can help mitigate this (for example storing
        extra bits of information about the closest free block).

        There are other techniques available, such as tracking free extents in
        a \bplustree like XFS \cite{XFS_scalability} or a proposed space map
        for EXT4 which relies on RB-trees \cite{ext4_space_maps}. Both of these
        approaches rely on space being managed in extents (see
        \ref{sec_block_size}).  Their major drawback of these approaches is
        their significantly increased complexity.

        In the end the classic bitmaps were selected. They work well enough,
        their implemntation is simple and a variety of allocation policies are
        possible with them. Alternative approaches depend on extents which are
        not supported and would introduce far too much extra complexity to the
        project for a limited benefit considering the generally basic design
        and time constraint. Time is well spend developing other features.

        Note that the free list isn't in the ilist, because of the bootstrap.

    \section{Superblock?}

        Say convention it's on block 1. It stores inode block nums to free list/ilist
        It's minimal and stores very little

        approaches rely on space being managed in extents. Their major drawback is
        the significantly increased complexity.

        In the end a file of the classic bitmap approach was selected. They
        work well enough, their implementation is simple and a variety of
        allocation policies are possible with them. Alternative approaches
        would introduce far too much extra complexity to the project for a
        limited benefit considering the generally basic design. Due to the time
        constraint, time is better spend developing other features.

    \section{The inode table and directories}

        Both structures are files containing a linear array of items. At the
        start of each file there is a small header which stores how many
        entries each file has. The only difference is in the items they store.
        The inode table contains inodes while directories store the usual
        (name, i-number) tuples. In both cases this is functionally equivalent
        to UFS (\ref{sec_UFS}).

        very different use cases.
        different arguments
        for simplicity keep linear
        made sense to unify

        Both structures serve very different purposes from a user facing
        perspective. However, the workload they need to complete is very
        similar. Both store a list of unordered entries. The directoy has to be
        searched to find the i-numbers of names. The inode table has to be
        searched to find a free entry to place a new inode. The only functional
        difference is that the inode table can be indexed to produce a result
        in constant time (and to that effect can be considered ordered, however
        the contents themselves have no particular order except their indices).

        As this filesystem strives for simplicity (\ref{sec_methodology}) it
        makes sense to unify the two structures. There exist bespoke
        implementations for both like ext4's Htrees for directories
        (\ref{sec_htree}). However, adding them introduces a lot of extra
        complexity which is undesiriable. Such features will not benefit the
        basic case much as the overwhelming majority of lookup in the inode
        table are i-number to inode resulutions (which are constant time as the
        index is reused in the table). Meanwhile, for home use, which this
        filesystem targets (\ref{sec_problem}), directories tend to not have a
        large number of files in them \cite{contents_strudy}. Therefore an
        imporovement in their speed will not be particularly noticable.

        As a result, a simple linear approach is adopted. It has pretty much
        the same benefits and the same drawbacks as those for the free list as
        the underlying structure is the same.

    \section{inode}

        Once again, to maintain simplicity, the inode contains only essentail
        information. This includes metadata like the owner's user and group
        IDs, the access permissions and creation/modifiaction timestamps.
        Non-metadata fields are the file's size and the head to the tree of
        data nodes, stored as a B-tree (see \ref{sec_design_btree}).

        The metadata is required to make a file understandable to Linux. It is
        also very simple to store as it has a known size and layout, is only
        stored in one place, and never referenced elsewehre. It could be
        reported as a fixed value (for example root as owner with full
        permissions for everything and the UNIX epoch as a creation timestamp)
        but omitting this information makes the filesystem unpleasant to debug,
        as it can be hard to tell if files with the exact same metadata are
        actually different or there is a bug.

        % TODO: should I include an strace trace of cat?
        The size and data fields are critical to implementing an inode and
        cannot be done away with. The data is obvious, since the file itself
        must be stored somewhere and the inode is the only place that
        references this. The need for the size is less apparent, since all
        systemcalls for manipulating a file, like \syscall{open},
        \syscall{read}, \syscall{write}, \syscall{lseek} and \syscall{close},
        do not reference it in any way. The data tree contains enough
        information for this value to not be stored explicitly.  For example,
        the tree will not contain a block for reading past the end of the file
        which will result in an error. However, basic programs like
        \monospace{cat} (which are one of the simplest way to access files
        \cite{TLDP_proc_access}) read the size of the file before making a
        request for its size (with \syscall{stat}) before reading it. This
        presents a challenge as, although it is possible to find out the size
        on each such call, it requires a nontrivial amount of code to do so.
        Needless to say, it is also unnecessarily inefficient. For the purpose
        of less code and a speed increase, it was deemed that storing this
        information and taking the slight complexity with it is worth it
        overall.

    \section{B-tree}
        \label{sec_design_btree}

        % TODO: chapter for FAT32 in backgroun? and double check exactly how it
        % did things to see if I said things right or if i'm subtly wrong
        Despite its ubiquity, the B-tree is not the only choice for storing
        data blocks for an inode. Simpler techniques have historically been
        used. Originally, UFS and FFS used a generic tree of staggered
        indirection (\ref{sec_UFS} and \ref{sec_FFS}). Another was FAT32 which
        used a linked list to describe all data blocks \cite{fat32}.

        The FAT32 scheme was experimented with, as it is quite a lot simpler
        than a full B-tree implementation. However, it proved very difficult to
        reason with on random accesses and performance was also abysmall
        (something FAT32 has always struggled with). Because of this it was
        decided to use a \bplustree from an external implementation. The B-tree
        is much more suited for this task (\ref{sec_btree}) and supports
        standard operations like insert, delete and lookup natively and
        performanlty.

    \section{Hardware considerations}
        \label{sec_hardware}

        % TODO: can reference ext4 data locality chapter to save on words
        A major focus for this filesystem was SSD instead of spinning
        mechanical hard drives. As a result, there are no major provisions for
        locality. This is because SSD random IO performance is almost in line
        with their sequential capabilities \cite{servethehome_review}. As a
        result, things are laid out in an arbitrary best fit fashion along the
        disk instead of considering future growth or access patterns.

        There is a single provision for increasing data locality. It is to have
        a minimum allocation size. This is because greater data locality "can
        increase the size of each transfer request while reducing the total
        number of requests" \cite{ext4_docs}. That minimum is set to 20
        blocks for a balance between high preallocation but not wasting much
        space.

    \section{Redundancy}

        The aim for the redundancy is "RAID with checksums", since RAID alone
        cannot be relied upon for data integrity (\ref{sec_RAID_problems}).

        First, to combat silent drive unreliability (see \ref{sec_reliability})
        all data must be checksummed. There are two choices for where this can
        happen: at the data structure level or at the block level. If done at
        the data structure level (i.e. every "pointer" would also include a
        checksum of its data), like EXT4 (\ref{sec_EXT4}, then there is a
        benefit that the data and its checksum are spatially separate (a
        provision ZFS has \ref{sec_ZFS}). However, this approach adds
        significant complexity for checks at all places. If done at the block
        level, then the rest of the filesystem can be ignorant about the
        existance of checksums and carry on as normal, reducing complexity and
        size requirements. Since this filesystem does not implement a tree of
        blocks, like ZFS (\ref{sec_ZFS}) does, this has the drawback that
        checksums are local to the data they protect. However, implemneting
        this is very trivial as it only needs to extend the block level
        interface and the entirity of the rest of the design can remain the
        same. This second approach is chosen for primarily this reason.

        For the question of where to store the checksum, there is only one
        solution that is not particularly involved. Storing the checksum within
        the block makes a corruption of the block extremely likely to corrupt
        the checksum too making it a very bad idea. The next best thing is
        having separate aggregate blocks that contain the checksums for as many
        blocks as possible just after them.

        Next comes the question of how much space to allocate for checksums.
        This is a space versus integrity tradeoff. On one hand the bigger the
        checksum the better since it allows for a lesser chance for collisions
        to mess things up. On the other hand, we want overhead to be as small
        as possible so that drives can be better utilised. For example, ext4
        allocates 4 bytes for each structure \cite{ext4_docs} but that is
        mostly for backwards compatibility and there simply not being room for
        more. ZFS stores 32 \cite{ZFS_docs}[basic\_cocepts/chechsums] for each
        block (usually 128KiB), which is estimated to be around 0.5\%
        \cite{ZFS_overhead}.

        For the scheme above of collating checksums for blocks in a separate
        block, a 4 byte checksum requires about 0.1\%, 16 bytes are 0.39\% and
        32 bytes are 0.78\%. A 0.1\% overhead would be idea, however this
        limits the choice of checksum algorithm severely (mostly Cyclic
        Redundancy Checks (CRCs) and Fletcher checksums
        \cite{embedded_checksums}). 64 bytes and more would be better for
        integrity but the overhead reaches single digit percent overhead which
        becomes significant. 32 bytes is the largest amount that is not
        significant enough to raise eyebrows. To make block numbers easy to
        compute this will mean every 128th block contains the checksums for the
        previous 127. The 128th place will be technically unused, so a checksum
        of the block itself for a bit of added integrity.

        Now, since checksums and data will be local to each other, the checksum
        algorithm must be as resiliant as possible to minimise the chance that
        a data corruption results in a checksum collision and the data is
        reported as unaltered. Even though the block checksums itself and that
        checksum is stored on the same block next to all the others, a corrupt
        block could feasibly alter an arbitrary checksum and the metachecksum
        at the same time. Since we are targetting SSDs (\ref{sec_SSD}) we may
        align checksum characteristics with SSD failure patterns.

        One way to measure SSD failures is with the raw bit error rate (RBER).
        The RBER is defined as the number of currupted bits per the number of
        total bits read \cite{flash_reliability}.
        \citeauthor{flash_reliability} take measurements over a vast number of
        drives with single-level cell (SLC, 1 bit per memory cell) and
        multi-level cells (MLC, 2 bits). They measure RBERs between $10^{-8}$
        and $10^{-5}$. \citeauthor{bit_error_mlc} finds similar rates for MLC.
        On the other hand \citeauthor{bit_error_qlc} finds that triple-level
        cells (TLC) and quad-level cell (QLC) suffer RBER rates of around
        $10^{-3}$ and $10^{-2}$ respectively. Although
        \citeauthor{flash_reliability} also note that RBER rates are a poor
        indicator of unrecoverable errors it puts an upper bound on their
        likelyhood. Also all of them (\cite{flash_reliability},
        \cite{bit_error_mlc}, \cite{bit_error_qlc}) cite error correcting codes
        to mitigate these errors we can expect much lower rates that would
        trigger our checksum recovery mechanisms.

        % TODO: that facebook large scale study idk if cited here. Check i've cited it right

        From the above we can expect a worst case failure of rate of a few bits
        per block. As \citeauthor{flash_reliability} and
        \citeauthor{flash_large_scale} note, these failures are not uniform and
        change with SSD age and wear. With further analysis from
        \citeauthor{flash_error_manual} it is apparent that this is due to a
        wide range of physical phenomena. However, it becomes apparent that
        these errors are relatively isolated from each other where only
        adjacent cells can affect each other due to the way cells store
        electrons. Further, due to effective error correcting techniques actual
        unrecoverable errors are much lower than the RBER.

        There are many ways to detect errors. Ext4 uses CRC \cite{ext4_docs}[checksums]
        while ZFS uses sha256 \cite{ZFS_docs}[checksum]. Other examples include
        Fletcher, Adler checksums \cite{embedded_checksums} and
        Bose–Chaudhuri–Hocquenghem (BCH) \cite{flash_error_manual}. We settle
        on hash based checksums for several reasons. First, we rely on extra
        drives to provide redundancy for error correction (to be described
        next) so a BCH is unnecessary. Then Adler and Fletcher checksums
        perform worse than good CRCs for a given number of bit errors
        \citeauthor{embedded_checksums}. Finally, the most common CRCs (eg.
        CRC32) produce short check sequences (32 bits for CRC32). A best case
        scenario will have a probability of undetected errors of
        $\frac{1}{2^k}$
        % BAD sentence (bad until end of paragraph tbh)
        where k is the length of the checksum \cite{embedded_checksums} and a
        longer checksum simply yields a lower probability. While CRC64 and
        CRC128 do exist they are not typically implemented in hardware so are
        unlikely to yield good speed. This leaves hashes as the most viable
        option, especially since they strive for similar inputs to produce
        different outputs, which is exactly the use case we expect.

        % Cite where I got the SHAs from?
        This leaves the choice of hash. Older hashes like MD5 and SHA-1 are
        probably sufficent but they are starting to show their age, for example
        SHA-1 is starting to have known collisions \cite{SHA_collision}.
        Besides, we have allocated more space than the 160 bits for SHA-1 so we
        can use the extra security from more modern ciphers. The major
        candidates are SHA-2 (SHA256) and SHA-3 (SHA3256). Ideally, we would
        use the latest and gratest SHA-3, however in the interest of efficiency
        we select SHA-2. This is because on modern architectures it is around 3
        times faster \cite{hash_stats} and the application we use it for is
        very unlinkely to experience malicious bit manipulation so the security
        is sufficent.

        % TODO: have the checksum be outside the block. Modify the allocator
        % somehow. OOOOO you can even checksum the checksum blocks too (might
        % be a good idea as it might not be too much memory?)

        Then, for the choice of RAID level, there are many options, from the
        classic RAID levels 1 thorough 6, to some nonstandard and exotic
        versions like RAID-Z (\ref{sec_RAID}). All have various performance
        profiles and offer protection against various number of drive failures
        However, for all intents and purposes they are interchangable - no
        matter the level, the result is a contiguous pool of redundant storage.
        For this reason, as a proof of concept, RAID 1 is selected. A basic
        mirror provides just enough redundancy while being simple to implement.
        There are no fancy stripe or parity calculations. One of the two disks
        can fail outright and all data will still be intact.  In line with the
        home user target, two drives is the smallest possible redundant number
        of drives and most likely for the reason of cost and space. Any more
        guarantees are not strictly necessary and if required can be added at
        little cost.

        % TODO: test these scenarios, add checks for them Also, do some mathys
        % analysis of the likelyhood: say 10^-5 of a block corupt, 4 blocks
        % need to corrupt and then a collision too should be like 10^-20 or
        % some shit
        Now, even though data and checksums are relatively local, this is
        compensated for by the RIAD arrangement. In the unlikely event that a
        block gets currupted and its associated checksum block gets corrupted
        too in such a way as to cause a collision and appear to be correct,
        both of these blocks have a mirrored copy. Their checksums can be
        compared to see if they are identical. The odds of two independent
        blocks corrupting in such a way as to cause a hash collision and doing
        so twice on two independent drives is extremely low.
