\chapter{Discussion and Conclusion}

    \section{the perfomance}
    \section{the implementation}
    the desing? idk what to call this

        Modern filesystem's complexity is justified and the performance
        analaysys (\ref{sec_perf} ?) confirms that. Each seemingly small
        addition has a large impact on some workload. There is no silver bullet
        feature that magically makes a filesystem fast. For this reason the
        performance of the proposed filesystem can never match or exceed that
        of any established filesystem. However, there are three features that
        can bring it most of the way there. First, is the use of B-trees. As we
        have already seen (\ref{sec_btree}) they are well established and the
        performance of this filesystem reflects that on sequentail reads
        (everywhere else there are at least two on top of each other or
        something else slowing it down (\ref{sec_perf}).

        Second, is the use of extents which this filesystem does not implement.
        Their need is most apparent for allocations and sequentail accesses but
        impact every aspect of a filesystem where large chunks of data are
        stored. Modern workloads and filesises measured in the giga and
        terabytes cannot be accommodated with individually addressed data
        blocks. The logarithmic complexity of any underlying storage method
        (like the B-tree) is dwarfed when a linear or a large polynomial of
        accesses are required.

        Third, and perhaps most importantly is rigorous engineering effort into
        integrating all features. While this is not a feature per se, but
        allows for all other features to be utilised properly. Due to the time
        experimental nature of this project this was not emphasised and
        performance suffers accordingly especially in sequentail accesses.

        With this in mind, the myriad of features and tens of thousands of
        lines of code in ext4 fall into place. A modern performant filesystem
        cannot be simple by design nor by implementation and a simple
        filesystem cannot be performant.

        % say it is feasible to improve it to go most of the way - just would need quote a bit of time and 10k LOC

    \section{Did this solve the problem? (Outcome)}
    % more like the reliability

        Despite the performance of this filesystem, its main task - data
        reliability - performs well. It is not ironclad, like ZFS, but is not
        meant to be. It is robust and provides a good degree of integrity. It
        is many orders of magnitude better than the unsecured filesystems in
        use today. It also provides the safety of RAID without a lengthy
        cost-benefit analysis of different techniques to mitigate the various
        problems and the drawbacks that might come with those mitifations.

        For laptops this filesystem is of limited use. The tendency for low
        profile and minimal devices makes adding an extra drive cost
        prohibitive. Nevertheless, the checksumming can still be utilised as a
        data integrity guarantee and the redundancy can be provided by an off
        site backup. New developments, like Apple's M1 Ultra chips with
        on-board flash controllers
        \cite{https://www.notebookcheck.net/Mac-Studio-SSD-does-not-work-on-NVMe-top-level-ARM64-SSD-controller-in-M1-Ultra-makes-it-nearly-impossible-to-swap-out-or-add-raw-storage-modules.609363.0.html}
        can make this setup viable with a single flash board.

        For home and office desktop computers this filesystem makes a lot of
        sense. These machines are bulky enough to house several drives (and
        often do) and especially for office use the extra integrity can be
        useful in missin critical environments.

        Finally, for home NAS setups, this filesystem is an obvious choice.
        their long term storage necessitates data integrity guarantees. A RAID
        setup is flawed while ZFS would be hard to administer.

        % TODO: explain about the mirror instead of a 3 drive arrangement (1 parity, 2 usable)

    % TODO: i need to figure out what my point with this is
    \section{Difficulties in modern operating system implementation}

        Modern operating systems are enromous projects. Just the latest release
        of Linux, 5.17, included over 13000 patchsets by about 2000 developers
        \cite{Linux_dev_count}.  Specifically for EXT4, there were almost 4000
        lines of insertions and deletions by 22 people over 74 commits
        (command: \monospace{git log --numstat v5.16..v5.17 fs/ext4}) just for
        its maintanence. Considering that releases are spaced 2 to 3 months
        apart \cite{Linux_dev_process}, EXT4 has existed since 2006
        \cite{ext4_origin} and that it is about 60000 lines of code (cmd:
        \monospace{cloc fs/ext4}, EXT4 represents many person-decades of work.
        As a comparison, this project is about 2500 lines of code spanning the
        time of about two releases by a single author. That is to say, any
        sizeable developments in operating systems these days are big efforts
        by teams of multiple teams over large timespans frequently backed by
        big organisations.

        Despite this project's small size and apparent uphill battle, it is
        still a notable development. Modern tools, like FUSE (\ref{sec_FUSE}),
        have gone a long way to simplify development and take it out of the
        kernel whenever it is not necessary. One notable example is SSHFS
        \cite{SSHFS} which is particularly useful
        tool largely maintaned by 2 developers.

        Even though this project is far from a full kernel driver, which a
        truly successful implementation would be, it is notprevented from
        becoming a fully functional filesystem.

    \section{Unexpected benefits of the methodology}

        The methodology (\ref{sec_methodology}) greatly benefited this project.
        The aim for an always working prototype lead for many issues to be
        timely discovered and desgin altered before too much momentum had been
        built up. There were two notable examples for this: the everything is a
        file phylosophy (\ref{sec_files}) and the linked list method for
        storing file data (\ref{sec_btree}??????????).

        Storing all metadata as a file presented an interesting circular
        problem: the routine to write a file depended on a routine to extend a
        file when writing past its end, but to allocate space for a file the
        free list (another file) needed to be written by the same routine.
        Small bugs frequently resulted in infinite recursion which is slow to
        debug. Catching this flaw early allowed for design to be modified to
        reduce the problem.  Some of the changes were keeping the free list out
        of the inode table, making its access direct instead of thorugh another
        level of recursion.  Another change was storing the block number on
        which an inode is stored in the inode itself to allow for updating its
        size to happen directly.

        The linked list method of storing data blocks was particulary
        troublesome. Although, the reason for this is still not entirely clear,
        reasoning with its block layout and performing calculations was very
        difficult and error prone. Seemingly simple operations like calculating
        which sequential block an arbitrary byte would reside in was a big
        routine with small edge cases that lead to duplicate code. Luckily,
        this prompted this approach to be abandoned and replaced with a B-tree
        (\ref{sec_btree}). Perhaps largely owing to the complete
        implementation, the project's complexity dropped substantially and
        progress was much quicker.

    \section{Conclusions}

    \section{Ideas for future work}

        This project is far from a production ready filesystem. There are many
        features that have been left out and some things that can be considered
        as they may be suitable for this project.

        The most apparent one is full coverage of all operations that FUSE
        exposes. For a truly general purpose and user friendly implementation,
        it must support any use case thrown at it.

        Then, in the interest of time, some code is not in a desirable state.
        Examples include some duplication between the directory and inode table
        code, block "pointers" use a generic type where they should ideally
        have one of their own even if just for semantics and general
        optimisation improvements as fixed data structures are repeatedly read
        where this isn't strictly necessary. Memory management could also be
        much better to reduce the programs' memory footprint. While on the
        topic of code, the B-tree implementation's use of pointer has been
        hijacked to carry non-memory related values. Converting the types to a
        new descriptive one would be a great improvement to the code.

        ZFS's tree of blocks (\ref{sec_ZFS}) is also a prime candidate for
        inclusion. The current impelementation has checksums of all blocks,
        however it does not have checksums of checksums. ZFS does this in a
        manner very similar to a blockchain and is a very elegant solution to
        the integrity problem. It has some performance implications, but
        adapting it here may lead to big integrity guarantee improvements.

        Notable omissions from the project are concurrent access and caching
        support. Finding a way to include them would greatly improve
        performance and is practically a must for a real filesystem.

        Reliability detection - too many errors will cause the drive to be expunged

        the performance lol

    \section{Choice of userspace?}
