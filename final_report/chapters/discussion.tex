\chapter{Discussion and Conclusion}

    \section{The implementation}
        \label{sec:implementation_discussion}

        Modern filesystems' complexity is very high and the performance
        analaysys (\autoref{sec:benchmark} and \autoref{sec:perf}) show that it is
        justified. Each seemingly small addition has a large impact on some
        workload. Provided enough such features are included, most workloads
        are efficient. There is no silver bullet feature that magically makes a
        filesystem fast. This is why the performance of the proposed filesystem
        can never match or exceed that of any established filesystem. However,
        there are three features that cover a wide range of workloads and can
        bring performance most of the way there.

        First is the use of B-trees. As we have already seen (\autoref{sec:btree})
        they are well established and the sequentail read performance of this
        filesystem shows it is competitive.

        Second is the use of extents, which have been included as far back as
        \cyear{LFS}. This filesystem does not include them under the
        assumption that SSDs can handle lots of random accesses well (discussed
        in \autoref{sec:block_size}). This, however, introduces an unforseen
        problem. Not collating chunks of storage means each one must be indexed
        independently. The index will not be large per gigabyte (assuming 8
        byte pointers, the \numprint{200000} blocks (\autoref{sec:perf}) require
        about 2 MiB which arranged in a B-tree with 4096 byte nodes will only
        need one extra node to complete the tree) its space overhead is
        neglegible so it can be ignored. However, managing a tree of this size
        amplifies the number of operations required. An allocation for a
        gigabyte of space becomes \numprint{200000} node insertions into the
        tree for example. This is an exponential increase over an even badly
        fragmented filesystem that can only allocate extents of single digit
        megabytes (and it that case the tree will need about a thousand
        entries).

        As a result, the speed of any one operation is not slower - drive
        utilisation is 100\% and time is not wasted due to the medium. However,
        the complexity of operations increases in orderds of magnitude. A
        logarithmic complexity of any underlying data structure (like the
        B-tree) is dwarfed when a linear or a large polynomial of accesses are
        required. Having this much processing required reduces observed
        efficiency. By \cauthor{FFS}'s measurements this arrangement would
        be optimal, however performance will be wasted.

        On second review, this reality seems well understood. The literature's
        historical focus on mechanical hard drives and their seek inefficiency
        means that the justification for extents is based on ageing factors.
        Even though those factors have evidently changed with SSDs, the
        solution remains valid and has not sparked a need to reconsider them.
        In the meantime, engineers appear well versed in this fact, however,
        their knowdledge has a tendency of not being externally documented
        (indeed, ext4 only has a single sentence that alludes to that fact) and
        being integrated into the large code base. In any case, the mere
        existance of multi gigabyte workloads (like videos for home users) make
        this feature a essential for any filesystem.

        The third and most subtle feature is a polished implementation.  to
        integrate all features. While this is not a feature as such, it allows
        for all other features to be utilised properly. Due to the focus on
        data integrity and the experimental nature of this project, this was
        not emphasised and performance suffers accordingly. This prototype
        successfully demonstrates data integrity however as a prototype is has
        not production readiness.

        With this in mind, the myriad of features and tens of thousands of
        lines of code in \nameref{sec:ext4} fall into place. It is a huge
        project developed by many people over a very long time. A modern
        performant filesystem cannot be simple by design nor by implementation
        and a simple filesystem cannot be performant. The times of
        \nameref{sec:UFS} are long gone and modern computer systems are
        inherently complex.

    \section{Did this solve the problem? (Outcome)}
    % more like the reliability

        % TODO: discuss simplicty vs outcome

        % omits things like the chaining of checksums
        Despite the performance of this filesystem, its main task - data
        reliability - performs well. It is not ironclad, like ZFS, as it does
        not implement every possible feature and does not actively protect
        against every eventuality. It is, however, also not meant to be.
        Covering every risk under the sun introduces extra complexity and
        configuration considerations which are simply overkill in a home
        setting. This is especially true as sensitive data should have
        additional data regardless of redundancy \cite{Backblaze_321}. Getting
        a robust implementation that gets most of the way is sufficent. This
        filesystem does exactly that. Having data integrity guarantees is many
        orders of magnitude better than current filesystems' reliance on the
        questionable track record of SSDs, if for no other reason than peace of
        mind. It also provides the safety of RAID without a lengthy
        cost-benefit analysis of different techniques to mitigate the various
        problems and the drawbacks that might come with those mitifations
        (\autoref{sec:RAID_problems}).

        For laptops this filesystem is of limited use. The trend towards low
        profile and minimalistic devices makes adding an extra drive cost and
        space prohibitive. Nevertheless, the checksumming can still be utilised
        as a data integrity guarantee without any redundancy. New developments,
        like Apple's M1 Ultra chips with on-board flash controllers
        \cite{m1_ultra_controller} can make this setup viable with a single
        flash board although this is an area for further research.

        For home and office desktop computers this filesystem makes a lot of
        sense. These machines are bulky enough to house several drives and
        accommodate any redundancy setup that may be desired. The lower
        administion overhead can help otherwise sceptical users to the
        complexity of solutions like \nameref{sec:ZFS} make the leap add some
        much needed integrity to their storage. Especially for office use the
        extra integrity can be useful in mission critical environments where
        corrupted files can cost many times more than the hardware requirement
        to prevent them.

        There has been a recent trend towards cloud based backup solutions like
        Google's Drive \cite{gdrive}, Apple's iCloud \cite{icloud} or ad-hoc
        services like Backblaze \cite{Backblaze_321} or Dropbox \cite{dropbox}
        instead of relying on any redundancies on the devices themselves. These
        solutions perform very similar integrity checks, hashing all files, and
        regularly uploading all specified directories to the cloud where full
        integrity and redundancy is guaranteed \cite{Backblaze_arch} with much
        higher degrees of security and usually also preserve file history.
        These solutions tend to have a low but monthly price (usually around
        £8 for 2TB).

        These are very good backup solutions, however their are of limited use
        as simple redundancy. Replacing misbehaving or outright failing drives
        may be isolated from data loss, however, it will certainly involve long
        down time. When it comes to pure integrity they are valid solutions as
        they integrate well with operating systems and the history features
        allow for recovery of silent data corruption. They are slightly
        inferior, however, as silent data corruption may have to be manually
        corrected even if all data is preserved. One major drawback though is
        the long-term price. The low monthly costs are unlikely to raise
        eyebrows but over longer periods they add up to a similar amounts of
        owning the drives. As a result, it is believed that these solutions are
        viable alternates but not direct competitiors to the proposed
        filesystem with user preference being the biggest factor.

    % TODO: you need more discussion. It's 40% of grade and it's like 6 pages
    % \section{Choice of userspace?}

    \section{Unexpected benefits of the methodology}

        The choice of methodology (\autoref{sec:methodology}) greatly benefited
        this project. The aim for an always working prototype lead to many
        issues being timely discovered and desgin altered before too much
        momentum had been built up. There were two notable examples for this:
        the everything is a file phylosophy (\autoref{sec:files}) and the linked
        list method for storing file data (\autoref{sec:design_btree}).

        Storing all metadata as a file presented an interesting circular
        problem: the routine to write a file depended on a routine to extend a
        file when writing past its end. However, to allocate space for a file
        the free list (another file) needed to be written by the same routine.
        Small bugs frequently resulted in infinite recursion which is slow to
        debug. Identifying this flaw early allowed for design to be modified to
        reduce the problem. Some other timely changes were keeping the free
        list out of the inode table, making its access direct instead of
        thorugh a level of recursive indirection. Another change was placing
        the block number on which an inode is stored in the inode itself to
        allow for updating its size to happen directly.

        The linked list method of storing data blocks was particulary
        troublesome. Despite an inital expectaion for simplicity, reasoning
        with its block layout and performing calculations was very difficult
        and error prone. Seemingly simple operations like calculating which
        sequential block an arbitrary byte would reside in was a big routine
        with small edge cases that lead to duplicate code. Luckily, this
        prompted this approach to be abandoned and replaced with a B-tree.
        Perhaps largely owing to its complete implementation borrowed from
        Linux, the project's complexity dropped substantially and progress was
        much quicker.

        Had a more upfront method been used for this project (predefined
        targets for example) it would have failed. Its high complexity and wide
        field of factors for considerations meant any preset plan could crumble
        at any moment. This way a general goal was set and the project
        naturally evolved to what it is now.

    \section{Conclusions}

        This project has produced a functional Linux filesystem design and
        prototype. The main feature of the filesystem is its strong integrity
        guarantee. This is achieved with full built-in redundancy with robust
        checksumming usging modern cryptographically secure hashes of every
        block. It is targetted towards SSD deployments and takes into account
        their unique reliability characteristics. The filesystem itself is
        enabled by a portable FUSE module for use accross a wide range of Linux
        distributions. It integrates standard filesystem featues like its use
        of B-trees for data indexing.

        The design has a flaw that drastically hampers performance. Under an
        assumption of SSD speed, it does not include extents. However this
        choice avalanches simple operations into complex problems. As a result
        most aspects of the filesystem perform an order of magnitude lower than
        desired. Nevertheless, due to the robust design, changing this choice
        does not affect other parts of the design. The implementation can be
        accordingly modified without much reworks.

        The proposed solution solves the problem of data integrity to a
        satisfactory degree for home use. As a result it is a good potential
        replacement to the common ext4. It is substantially simpler and lacks
        the plethora of options of ZFS so it may be considered in places where
        such a filesystem is desired but not the accompanying complexity.
        Finally, the targetted deployments for it are cost competitive with
        cloud based backup solutions although those target somewhat different
        class of usage. As a result home users gain a useful tool which may
        better fit their needs.

    \section{Ideas for future work}

        This project is very minimal and not ready for production use. Many
        features have deliberately not been included and improvements can be
        made accross the board. There are features that may be suitable for
        inclusion but would need further consideration. It also opens some
        interesting topic for consideration.

        First, the filesystem can have the shortcomings discussed in
        \autoref{sec:implementation_discussion} addressed. Namely, extents can be
        added and access routines optimised to not split big accesses into
        small ones. Some basic caching in those routines can greatly benefit
        performance too.

        As with most project, general code improvements would always be
        welocme. Notably missing are new generic types to improve data type
        semantics, some customisations to the B-tree implementation to better
        suit the particular need of the project, memeory usage optimisations
        and multithreading support will go a long way for production readiness.

        Then, less clear-cut features can be considered. Notably,
        \nameref{sec:ZFS}'s tree of blocks can be considered for additional
        security. Also, for single-drive systems, on-disk redundancy (in a RAID
        5 like arrangement) or sophisticated multi-bit error correcting codes
        may be more suitable for that segment. These can protect against
        fatally corrupt sectors while the drive is still functional for
        recovery.

        % things like trees in the free list and more complex allocator, trees in dirs, etc

        Finally, as noted in \autoref{sec:implementation_discussion}, further
        research into extents on SSDs is possible. They are evidently suitable
        for such a storage medium but were not developed for it. A better
        understanding of why this is the case can help filesystem developers
        optimise their designs and possible adopt new techniques.
