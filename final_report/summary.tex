Over the years UNIX like operating systems and their filesystems have gone a
long way. A large number of well thought out and robust features achieve very
high degrees of resource utilisation. Throughout, a major driver for many of them has
been drive unreliability. Drives routinely fail but even before that they
may misbehave in an undetectable manner. To ensure data integrity there exist
many solutions with various approaches. Notably ext4 ensures filesystem
consistency thorough checksumming metadata. Redundant arrays of disks (RAID)
replicate the whole filesystem in case a drive fails. A notorious example is
ZFS which considers every eventuality and implements mitigations for it.

All of these solutions have problems though. Ext4 does not checksum data and
relies the drive will report it correctly, RAID has a write hole issue which
may corrupt data and ZFS's support for all its features comes at the cost of
great complexity. All the various approaches with their differing drawbacks
present a challenge for less technically savvy home users. They will not be
aware of the pros and cons of each or may be unable to administer them
appropriately to achieve their goals as a result.

To that effect, this project proposes a new filesystem that collects the good
things of each of these solutions while mitigating the major issues and making
it easy to use. It integrates modern techniques for ensuring data integrity
without compromising on performance by design. It also targets the ever so
ubiquitous solid state drives to optimise for what is most likely to be in the
target field of laptops, office desktops and home servers.

The proof of concept implementation achieves very good integrity. All data is
stored with full redundancy to protect against any pattern of drive
misbehaviour although greater focus is placed on SSD specific failures. Any
silent data corruption can be detected with secure hashes of all blocks and
restored from redundant copies. The result is cost competitive with a wide
variety of other solutions and is easy to administer. As such, it is a valid
choice for its target audience.

The final design deliberately avoids the use of extents which turn out to be
crucial for modern filesystems. Unsurprisingly, performance is suboptimal and
is about an order of magnitude lower than desired. As this deficiency is
unexpected its cause is investigated in depth which reveals a widespread but
incomplete understanding of the issue.

Nevertheless, the general design is robust and holds up its integrity promises.
It is implemented as a FUSE module to ensure portability for widespread and
easy use. The deficiency is isolated and does not require major reworks to
become production. As a result, this can easily be addressed and with some optimisations
the prototype can become suitable for general production use.
