%
% Motivation / Introduction
%
\section{Introduction}
    % <A brief overview of your project suitable for a non-specialised
    % audience. You may mention background and motivation, but should avoid
    % technical terms as far as possible. 1-2 paragraphs should be sufficient.
    % Here is an example of how to cite a reference \cite{parikh1980adaptive}.>




%
% Aims, Objectives and deliverables.
%
\section{Aims, Objectives and Delievarables}

\subsection{Project Aims}
    % <A brief summary of the primary aims of this project. Typically 1-3
    % sentences.>

    This project aims to produce a new redundant filesystem for Linux. It is
    intended to be a more light-weight alternative to the enterprise-ready ZFS
    and to address some of its shortcomings for home use while having most of
    its benefits.

\subsection{Objectives}
    % note this means `learn how to\ldots', `research into\ldots` are {\em not}
    % objectives, as they are intermediate milestones rather than final goals.
    % All objectives should be measurable, {\em i.e.} it should be possible to
    % provide evidence to confirm whether or not they have been achieved. 3-5
    % objectives is typical.>

    % TODO: fix citations
    \begin{description}
        \item[A modern filesystem design]

            The first and most important objective is creating a modern
            filesystem. Modern in the sense that it should be have comparable
            speed to other widespread filesystems, like ext4 and NTFS, and
            provide the most important features users have come to expect.

            Currently, this is expected to include standard inodes organised in
            some B-tree structure of blocks on disk. Blocks will be overwriting
            (instead of copy on write) and will only consider SSDs.

            A major goal is to not depend on an fsck utility. Crashes should be
            automatically detected and corrected on mount.

        \item[A filesystem implementation]

            The design will be implemented as a userfs module for Linux. It
            should be mountable as any other and seemlessly usable.

        \item[Redundancy and silent error correction]

            Since the filesystem will assume flash-based underlying media,
            which is prone to quite a few silent errors despite all the error
            correction features \cite{facebook_flash, google_flash}, the
            filesystem design will make according considerations about speed
            and reliability. It will integrate the redundancy features of RAID
            and provide assurances against data corruption, like block/file
            checksumming, which plain RAID does not. Silent error detection and
            recovery is the main focus for this filesystem.

            The end result should resemble ZFS in spirit, but without all of
            the enterprise considerations. That means fewer system resource
            requirements, less rigirous protection against crashes and higher
            device flexibility, if possible.

        \item[Comparison and evaluation]

            It is expected that the proposed filesystem will offer some
            benefits (at lest for home use) without taking steps back on
            usability and performance. To verify this, extensive testing will
            have to be conducted and the result compared against current
            solutions.

            The goal is the evaluation to drive the design to an extent so that
            the end result does not regress on expected features.

    \end{description}


\subsection{Deliverables}
    % <A list of what you will hand in at the end of the project. This will
    % include the final report (possibly spread across multiple deliverables,
    % if that makes sense for your project), code (possibly more than one
    % version), and so on. Ideally the deliverables should be cross-referenced
    % to the objectives. 2-3 deliverables is typical, but there can be more
    % depending on the nature of the project.>

    There are only 2: the final report and source code. The report will contain
    all background research, feature considerations and the basis for the major
    proposals. I expect this to include lots of graphs and drawings to
    illustrate choices, as they are quite convoluted. The source code will be a
    working userfs module which can be compiled and loaded into a modern Linux
    operating system and the filesystem immediately used.

%
% Plan
%
\section{Project Plan}
    % submission of the final report. This should discuss the key stages of
    % your project.>

    As of writing this report, most preliminary research has been completed.
    Next, are \ref{l1_num} major stges:


    \begin{enumerate}

        \item Look into B-trees and the layout of ext4/NTFS to get a feel for
            what current filesystems look like. This is not intended to be an
            in-depth look, but rather a guide on how to proceed.

        \item Immediately after that, research and implement the userfs module.
            This will include a basic layout on disk with a good ordering
            arrangement. Any housekeeping should be taken care of at this stage
            (mounting/unmounting procedures, system calls, etc) and files
            should be readable and writeable to a disk. A test environment
            should emerge in the process. The current idea is that this be done
            with loopback devices on my development machine.

        \item At this stage testing should be considered, especially failure
            simulation.  Since the major topic of the project deals with
            handling various failures, which are rather unpredictable
            \cite{google_flash}, there should be some mechanism available to
            insert them reliabily. It is likely that this will require the test
            environment be in a virtual machine (QEMU) which will be done at
            this stage if necessary.

            At this point a working, but very barebones prototype should be
            ready. From here until the end of the project the prototype should
            have a consistently working state which should go on until time
            runs out.

        \item Next is where the bulk of the work is expected to happen. There
            are only two mandatory parts to implement: redundancy  -
            replication at some granularity like RAID - and reliability -
            checksumming like ZFS'. If time allows, more features will be added
            to make the comparison to existing filesystems more accurate. At
            the end of various increments some testing criteria will be devised
            to assess the viability of the features and make to make changes
            accordingly.

        \item Finally, although the intention is to write up results for the
            report throughout the project, the bulk of it will happen at the
            end so some judgement will be exercise to cease work, even if
            incomplete, and draw up results for the final report.

    \label{l1_num}
    \end{enumerate}


\subsection{Timeline}
    % <A graphical description of your plan, often as a Gantt chart.>

    % To include a figure, uncomment and modify the following. Vary the 'width'
    % field until it looks suitable. This uses the 'graphicx' module that has
    % already been included in 'config.tex'.
    %\begin{figure}[htbp]
    %	\centerline{
    %		\includegraphics[width=12cm]{<Filename>}
    %	}
    %	\caption{<Caption text>}
    %\end{figure}



%
% Risk mitigation.
%
\section{Risk Mitigation}
    % <Identify risks to your project, and what you would do if they arose.>

    This project has a number of risks. It is quite involved and requires a lot
    of new skills. Despite that, it is believed that it is achievable with some
    alterations to the ideal vision for it:

    \begin{description}

        \item[Complexity of implementation] The biggest risk to this project is
            running out of time due to overwhelming code complexity from
            writing a filesystem driver. Kernel programming is a very elaborate
            exercise and due to a lack of sufficient experience in it, the
            implementation will be done in userspace as a userfs module. This
            certainly diminishes the efficiency of the implementation and may
            skew some performance measurements. However, userspace programming
            experience is plentiful and only has a small fraction of the things
            that could go (inexplicably) wrong. As the focus on of the project
            is not meant as an engineering feat so this should free up time for
            the more important academic aspect of it.

        \item[Incomplete prototype] Despite the reduction in complexity with a
            userfs, a large chunk of the work on the project is not towards
            making a better filesystem, but rather making a filesystem at all.
            To make sure the code is not completely unusable in case some
            feature fails, the project will use a sort of "rolling release"
            model. The idea is to implement and test features individually and
            in whole before starting a new one. If one cannot be implemented
            for whatever reason, the project can be submitted regardless.
            Anything after the initial barebones filesystem should be
            submittable so that even if the project ends up as a glorified
            literature review it can get as many marks as possible.

        \item[Full originality] This project will not attempt to create an
            original and filesystem with novel features. Due to the size and
            complexity of creating a filesystem this is not feasible for a
            final year project. Instead, since there are no originality marks,
            the novelty of the project is going to be very limited. None of the
            features are novel themselves, but the way they are arranged is not
            particularly widespread. This should drastically reduce the chance
            for unforseen problems and get the project off the ground faster.

        \item[SSD over HDD] Hard drives will not be considered. The whole
            filesystem will be implemented with the expectation that it will
            run on an SSD. This limit the pool of devices that have to be
            considered for any performance impacts.

    \end{description}

%
% Ethics.
%
\section{Ethics}
    % <If your project has ethical issues (e.g. gathering of user consent
    % forms), then you should state here how you intend to address them. If
    % there are no ethical issues then explicitly state: "There are no ethical
    % issues for this project.">

    This project has no ethical issues. It does not deal with humans in any way
    and only aims to better a set of tools which are widely accepted and
    available.
